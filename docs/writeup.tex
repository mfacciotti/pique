\documentclass{bioinfo} %\usepackage{amsmath} %\usepackage{enumerate}
\usepackage{graphicx} %\author{R.Y.~Neches \& E.G.~Wilbanks}
%\title{Analyzing ChIP-seq data with Pique}

%\addtolength{\oddsidemargin}{-.875in}
%\addtolength{\evensidemargin}{-.875in}
%\addtolength{\textwidth}{1.75in} %\addtolength{\topmargin}{-.875in}
%\addtolength{\textheight}{1.75in}

\copyrightyear{2011} \pubyear{2011}

% This writeup is intended to be submitted to Bioinformatics as an %
Applicaiton Note under either the Gene Expression or Sequence %
Analysis. These are the instructions to authors : % % Application Notes
(up to 2 pages; this is approx. 1300 words or 1000 % words plus one
figure) : Applications Notes are short descriptions % of novel software
or new algorithm implementations, databases and % network services (web
servers, and interfaces). Software or data % must be freely available to
non-commercial users. Availability and % Implementation must be clearly
stated in the article. Authors must % also ensure that the software is
available for a full TWO YEARS % following publication. Web services
must not require mandatory % registration by the user. Additional
Supplementary data can be % published online-only by the journal. This
supplementary material % should be referred to in the abstract of the
Application Note. If % describing software, the software should run
under nearly all % conditions on a wide range of machines. Web servers
should not be % browser specific. Application Notes must not describe
trivial % utilities, nor involve significant investment of time for the
user % to install. % % Software : If the manuscript describes new
software tools or the % implementation of novel algorithms the software
must be freely % available to non-commercial users at the time of
submission, and % appropriate test data should be made available.
Availability must be % clearly stated in the article. Authors must also
ensure that the % software and test data is available for a full TWO
YEARS following % publication. The editors of Bioinformatics encourage
authors to make % their source code available and, if possible, to
provide access % through an open source license (see www.opensource.org
for % examples). Authors should make every effort to use URLs that will
% remain stable. At the minimum, authors must provide one of: %
webserver, source code or binary. % %
http://www.oxfordjournals.org/our_journals/bioinformatics/

\begin{document} \firstpage{1}

\title[In a fit of pique]{Analyzing microbial ChIP-Seq data with Pique}
\author[Neches \textit{et~al}]{R.Y.~Neches\,$^{1,3}$\footnote{to whom
correspondence should be addressed}, E.G.~Wilbanks\,$^{1,3}$ and
M.T.~Facciotti\,$^{2,3}$ \address{$^{1}$Microbiology Graduate Group,
University of California, Davis.\\ $^{2}$Department of Biomedical
Engineering, University of California, Davis.\\$^{3}$Genome Center,
University of California, Davis.}}

\history{Received on XXXXX; revised on XXXXX; accepted on XXXXX}

\editor{Associate Editor: XXXXXXX}

\maketitle

%\newcommand{\imsize}{0.45\columnwidth}
%\newcommand{\threeup}{0.26\columnwidth}
%\newcommand{\cotwo}{$\text{CO}_{2}$} %\newcommand{\htwo}{$\text{H}_2$}
%\newcommand{\otwo}{$\text{O}_2$}
%\newcommand{\water}{$\text{H}_2\text{O}$}
%\newcommand{\htwos}{$\text{H}_2\text{S}$}

\begin{abstract} \section{Motivation:}

\noindent Most ChIP-Seq peak finders are designed to detect evidence of
protein-DNA binding events in eukaryotic datasets. To make
cost-effective use of current sequencing capacity, the peak finders must
be cleverly optimized to work with sparse-coverage data, and must take
into account the effect of chromatin structure on the variation in
background coverage. While numerous effective peak finders have been
developed for eukaryotic data, these algorithmic approaches can be
suprisingly error prone when run on high-coverage bacterial and archaeal
ChIP-Seq datasets.

% Why not? Examples, evidence, hypothesis.

\section{Results:}

\noindent Fortunately, many of the statistical challenges for peak
detection inherent in eukaryotic ChIP-seq data are not present in
bacterial and archaeal datasets; this is due in part to higher genome
coverage -- typically in inverse proportion to genome size -- and in
part to the absence of non-random coverage variation due to highly
structured chromatin.  In response, we have developed Pique, a
conceptually simple, easy to run ChIP-Seq peak finding pipeline for
bacterial and archaeal ChIP-Seq data. The software is cross-platform and
Open Source, and based on Open Source dependencies. Output is easily
imported into the Gaggle Genome Browser for manual curation of peaks and
the exploration of the dataset in the context of Gaggle-enabled
resources.

\section{Availability:}

\noindent The software is available under the BSD-3 license at

\href{http://github.com/ryneches/pique}{http://github.com/ryneches/pique
}.

\noindent A tutorial and test data are included with the documentation.

\section{Contact:} \href{ryneches@ucdavis.edu}{ryneches@ucdavis.edu}

\end{abstract}

\section{Introduction}

\noindent Next generation sequencing coupled with chromatin
immunoprecipitation (ChIP-Seq) is revolutionizing our abilty to
genomically map protein-DNA associations for a wide variety of proteins.
The growing popularity of ChIP-Seq has spurred the development of over
30 peak picking algorithms (for a nearly completely list see
\cite{wilbanks}). The relative performance represetative peak detection
algorithms on eukatyoric data and methods to assess performance have
been recently reviewed by several authors \cite{Pepke, Laajala_review,
too_many_peak_callers, peakranger, peak_benchmark}.  To make most
cost-effective use of current sequencing abilities the peak picking
methods have employed a number of sophisticated strategies to detect
peaks in the typically sparcely covered eukaryotic datasets for which
they are designed.

While ChIP-Seq has been predominantly used to interrogate protein-DNA
interactions in eukaryotic systems there are clear advantages to
adopting this technology for studying microbial systems that are largely
associated with the relatively small sizes of microbial genomes (the
genome of {\em E. coli} is $\approx$2000 times smaller than the human
genome). Eukaryotic ChIP-Seq necessarily involves more challenging
biochemical and statistical approaches than microbial ChIP-Seq, and so
we were surprised to find that software that works well in eukaryotic
systems does not perform adequately when presented with less challenging
data.

To our knowledge, only one other peak finding package, CSDeconv
\cite{CSDeconv} has been explicitly developed for finding peaks in
microbial ChIP-Seq data. This MATLAB package successfully finds peaks in
microbial ChIP-Seq data, but its application is limited by its
dependency on costly proprietary software, very slow performance, lack
of support for manual curation, and high false negative rate.

Herein we describe Pique, a conceptually simple, Python-based peak
finding package that enables easy and rapid peak finding in bacterial
and archaeal ChIP-Seq datasets.  The output is easily imported into the
Gaggle Genome Browser \cite{ggb} to enable rapid manual curation and
analysis of ChIP-Seq data in the context of other Gaggle-enabled
\cite{gaggle} resources (browsers are that can import GFF data also
supported).

Pique is also designed for use in systems have genomic complexities such
as IS elements, gene dosage polymorphisms and accessory genomes that
cause variations in sequence coverage unrelated to ChIP, or in cases
where the organism under study is not identical to the reference genome.
The resulting enrichment ``pedestals'' and ``holes'' can be very
problematic for detecting peaks and calculating enrichment levels. Pique
allows the user to optionally supply a map of these features as a GFF
file, and the software will automatically perform a segmented analysis.
(I LIKE!)

\section{Approach}

\noindent ChIP-Seq in bacteria and archaea yields coverage several
orders of magnitude larger than in eukaryotic systems.  This generates
data with near-continuous signal across the microbial chromosome rather
than the sparse coverage typically present in eukaryotic ChIP-Seq data.
This feature of microbial ChIP-Seq data permits simpler, faster
algorithms to be used. We have based our algorithm on classic noise
reduction techniques from signal processing.

Finally, given the importance of manual curation and settings
optimization, Pique has integrated curation support through the Gaggle
Genome Browser. This permits convenient interactive curation of the peak
list and analysis of the ChIP-Seq data in the context of other
Gaggle-enabled resources. Interactive curation of a microbial ChIP-Seq
data set can typically be completed in a few minutes.

\begin{methods} \section{Methods}

\noindent Prior to running Pique, 40-bp Illumina (Solexa) reads should
be quality filtered, quality trimmed, and aligned to a reference genome
using the user's preferred short-read sequence aligner. Pique requires a
BAM file as input \cite{sam_format}. We suggest using all contigs of the
reference genome as the mapping target, but the user may, if desired,
proceed with processing one contig at a time.

The user may optionally supply a map of coverage features. Pique
supports two modes; analysis regions and exclusion regions. By default,
Pique treats each contig as a single analysis regions, but the user may
designate regions within a contig for separate analysis. This is useful
where a gene dosage polymorphism has systematically altered the coverage
level in a large region. Exclusion regions are simply masked out of
their respective analysis regions, and are useful for removing coverage
variation due to repetitive DNA.

The user launches the primary analysis stage by providing alignment an
file for the ChIP data, an alignment file for the control data, and an
optional coverage feature map. The primary analysis proceeds thusly :

\begin{itemize}

\item The alignment files are digested, and the analysis regions are
initialized. If a coverage feature map provided, the analysis regions
are separated and the exclusion regions applied.

\item The coverage noise threshold is measured in both the ChIP and
control alignments by adaptive simulated annealing of the cutoff
threshold with respect to the peak recovery rate.

\item In each analysis region, the ``DC'' component is removed using
linear detrending. This removes effects due to coverage variation
features larger than about 100Kb.

\item High-$k$ noise in coverage is removed using a Wiener-Kolmogorov
filter. The filter delay $\alpha$ is chosen to approximate to the
expected footprint size of the targeted protein. The choice of filter
implies some specific assumptions about the nature of the coverage
noise. The Wiener-Kolmogorov filter was the first and simplest
statistical signal filter, first published by Norbert Wiener in 1949,
and independently derived in discrete-time form by Andrey Kolmogorov in
1941. The approach assumes the existence of two inputs; a ``true''
signal, and a noise source. Both are assumed to be stationary stochastic
processes combined additively. The choice of this filter minimizes the
assumptions.

\item A sliding window average is used to identify regions whose
coverage level deviates from the background. Peaks usually contain gaps
in coverage on the order of the experimentally selected fragment size;
the window width is chosen to correspond to this size. This yield simple
rectangular envelopes around putative regions of enrichment.

\item To determine if a putative enriched region corresponds to a
binding event, we require that the stop coordinate of the forward strand
enrichment envelope must fall between the coordinates of the reverse
strand enrichment envelope, and that the start coordinate of the reverse
strand enrichment envelope fall between the coordinates of the forward
strand enrichment envelope. (We call this the overlap criterion.)

\end{itemize}

% FIXME : The overlap criterion is a very crude version of the %
heuristic models other peak callers user. We should say something % here
about these models, and why decided not to use them.

\end{methods}

\begin{figure}[!tfbd peak - a nice one]%figure1
%\centerline{\includegraphics{fig01.eps}} \caption{HOW ABOUT A FIGURE OF
A PEAK with rectangular envelope and subsequent shape
envelope}\label{fig:01} \end{figure}

\begin{figure}[!tfbd data - reasonable spot showing curation]%figure2
%\centerline{\includegraphics{fig02.eps}} \caption{GAGGLE GENOME
BROWSER}\label{fig:02} \end{figure} \section{Discussion}

\noindent If a putative peak passes all of the tests above, Pique
concludes that the peak ``looks'' like a peak. To make sure that we are
not finding horsies by gazing at clouds (HUMM. . . ), we also require
that the integral of the coverage in the raw data within the putative
peak region exceeds the integral of the coverage in the background by a
margin set by the user. (Other tests for statistical significance may
also work, be more shiny, et cetera. For example, Monte Carlo
simulations of random subsamples of the ChIP track and the background
track until a coalescent is found. That would be fun!


\section{Conclusion}

We note that Pique should also work well with eukaryotic datasets
provided they are gathered with greater coverage than has been
previously reported.

NEED SOME SUMMARY SHOWING THAT THE PIQUE WORKS - THIS COULD BE A QUICK
PARAGRAPH SAYING HOW MANY PEAKS WE GET FROM THE BROAD DATA.

ALMOST DONE!

\section*{Acknowledgement} \paragraph{Funding\textcolon}

This project was funded by UC Davis startup funds to MTF, NSF graduate
fellowship AWARD NUMBER to EGW and DARPA AWARD NUMBER TO RN.

%\bibliographystyle{natbib} %\bibliographystyle{achemnat}
%\bibliographystyle{plainnat} %\bibliographystyle{abbrv}
%\bibliographystyle{bioinformatics}

\bibliographystyle{plain}

\bibliography{writeup}

\end{document}